\documentclass{llncs}
\usepackage[utf8]{inputenc}
\usepackage{amssymb}
\usepackage{amsmath}

\usepackage{llncsdoc}
% \usepackage{color}
% \everymath{\color{blue}}
%\everydisplay{\color{blue}}
\let\displaystyle\textstyle


\newcommand{\ie}[1] {
  \begin{itemize}
    #1
  \end{itemize}
}
% \usepackage{aaai}
\usepackage{times}
\usepackage{helvet}
\usepackage{amssymb}
\usepackage{amsmath}
\usepackage{courier}
\usepackage{multirow}
    \usepackage[bottom]{footmisc}
\usepackage{microtype}
\usepackage{tikz}
\usepackage{comment}
\usepackage{chngcntr}
\usepackage{float}
\usepackage[algo2e,ruled,linesnumbered,vlined]{algorithm2e}
\counterwithin{figure}{section}
\usepackage{algorithmic}
\usepackage{algorithm}
\renewcommand{\algorithmicrequire}{\textbf{Input:}}
\renewcommand{\algorithmicensure}{\textbf{Output:}}

% \usepackage[timestamp, dark]{draftcopy}
% \usepackage{doc}
\usepackage{url}
% \usepackage[sort]{natbib}
%\usepackage{amsmath,amssymb, amsthm}
\usepackage{balance}
% \usepackage[switch, pagewise, mathlines, displaymath]{lineno}

% \setcounter{secnumdepth}{2}
 \usepackage{fancyvrb}
\usepackage{graphicx}

\usepackage{listings}
%opening
\title{}
\author{}

\begin{document}



\section{Definitions}
\begin{definition}[Formula] \label{formula}
{\rm
\ie{
\item $\emptyset$ is a {\em formula}
\item \ $T \in D$, where T is a variable, a %\sparc\
ground term or an arithmetic term,
and $D$ is a set of %\sparc\
ground terms, is a {\em formula},
\item  $t_1\diamond t_2$, where $t_1$ and $t_2$ are terms and
$\diamond \in \{ = ,\neq,  \prec, \preceq\}$, is a {\em formula}, and
\item if $A$ and $B$ are formulas then ($A$ $\land$ $B$), ($A$ $\lor$ $B$), and  $\neg A$ are {\em formulas}.
}
}
\end{definition}

\begin{definition}[Empty Formula] \label{empty formula}
\begin{itemize}
 \item $\emptyset$ is an  {\em empty formula}
 \item if $A$ and $B$ are empty formulas then ($A$ $\land$ $B$), ($A$ $\lor$ $B$), and  $\neg A$ are {\em empty formulas}.
\end{itemize}
 \end{definition}
 In what follows  any empty formula is interpreted as \textbf{false}.
 
 
 
 \begin{definition}[Primitive Formula] \label{primitive formula}
  {\rm
\ie{
\item \ $T \in D$, where T is a variable, a %\sparc\
ground term or an arithmetic term,
and $D$ is a set of %\sparc\
ground terms, is a {\em primitive formula},
\item  $t_1\diamond t_2$, where $t_1$ and $t_2$ are terms and
$\diamond \in \{ = ,\neq,  \prec, \preceq\}$, is a {\em primitive formula}, and
\item if $A$ and $B$ are formulas then    $\neg A$  is a {\em  primitive formula}.
}
}
 \end{definition}
 


 \begin{definition}[Primitive Conjunction] \label{primitive conjunction}\\
 {\rm
  A formula $\mathcal{F}$ is  a \textit{primitive conjunction} if it is of the form $G_1 \land \dots \land G_n$, where $G_1,\dots,G_n$ are primitive formulas.
}
 \end{definition}
 
 \begin{definition}[Arithmetic Variable] \label{arithmetic variable} \\
  {\rm
    A variable $X$ occuring in a primitive conjunction $\mathcal{C}=G_1 \land \dots \land G_n$ is called arithmetic with respect to $\mathcal{C}$ if one of the following conditions holds:
    \begin{itemize}
      \item X occurs in an arithmetic term containing at least one arithmetic operation.
      \item one of $G_i$ is of the form  $X \in D$, where $D$ is a range of natural numbers.
      \item one of $G_i$ is of the form  $X \diamond Y$ or $Y \diamond X$, where Y is an arithmetic variable and $\diamond \in \{ \prec,<,\preceq,=\}$. 
   \end{itemize}
 
  }
 \end{definition}
\begin{definition}[Arithmetic Term] \label{aterm}
{\rm
 A term $t$ occuring in a primitive conjunction $\mathcal{C}=G_1 \land \dots \land G_n$ is called \textit{arithmetic} 
with respect to $\mathcal{C}$ if one of the following conditions holds:
 \begin{enumerate}
  \item $t$ is a number
  \item $t$ contains an arithmetic operation ('+','-', or '*').
  \item $t$ is not a record and all variables in $t$ are arithmetic with respect to $\mathcal{C}$
 \end{enumerate}
}
\end{definition}

 \begin{definition}[Primitive Arithmetic Constraint] \label{primitive arithmetic constraint}
{\rm
   A primitive constraint $G$ occurring in a primitive conjunction $\mathcal{C}=G_1 \land \dots \land G_n$  
is called \textit{arithmetic} with respect to $\mathcal{C}$ if one of the following conditions holds:
 \begin{itemize}
  \item $G$ is of the form $T \in D$, where $T$ is an arithmetic term with respect to $\mathcal{C}$ and $D$ is of the form $n1..n2$.
  \item $G$ is of the form $t_1\diamond t_2$, where both $t_1$ and $t_2$ are arithmetic terms with respect to $\mathcal{C}$. 
 \end{itemize}
}
 \end{definition}


\section{Algorithms}
\begin{algorithm2e}[H]\caption{ExpandSolve}
 \DontPrintSemicolon
 %\SetAlgoBlockMarkers{begin}{end}
 \KwIn{Formulas $\mathcal{F}$, ${\cal TODO}$, ${\cal C}$, 
         such that $\mathcal{F} \land {\cal TODO} \land {\cal C}$ is non-empty}
 \KwOut{\textbf{True} if $\mathcal{F} \land {\cal TODO} \land {\cal C}$ is satisfiable and \textbf{false} otherwise }
\If{ $\mathcal{F}=\emptyset$}
{
 \If{$\mathcal{TODO}=\emptyset$}
 {
   \Return Solve(Simplify($\mathcal{C}$),$maxint$) \;
 }
 \Else 
 {
   Let $\mathcal{TODO}$ be $G_1 \land \dots \land  G_n$ \;
   \Return {ExpandSolve($G_1$,$G_2 \land \dots \land G_n$,$\mathcal{C}$)}\;
 }
}
\ElseIf{$\mathcal{F}=\neg(A \land B)$} {
  ExpandSolve($\neg A \lor \neg B, \mathcal{TODO},\mathcal{C})$\;
}
\ElseIf{$\mathcal{F}=\neg(A \lor B)$} {
    ExpandSolve($\neg A \land \neg B, \mathcal{TODO},\mathcal{C})$\;
}
\ElseIf{$\mathcal{F}=A \lor B$} {
  \If{ExpandSolve($A,\mathcal{TODO},\mathcal{C})$=\textbf{false}}
  {
   \Return ExpandSolve($B,\mathcal{TODO},\mathcal{C}$) \;
  }
  \Else {
   \Return \textbf{true}\;
  }   
}
\ElseIf{$\mathcal{F}=A \land B$} {
  \Return ExpandSolve($A, B\land \mathcal{TODO},\mathcal{C}$) \;
}
\Else 
{
\Return ExpandSolve($\emptyset,\mathcal{TODO},\mathcal{F} \land \mathcal{C}$)
}
 \Return {\em true} \;
\end{algorithm2e}

\begin{algorithm2e}[H]\caption{Simplify}
 \DontPrintSemicolon
  \KwIn{Primitive conjunction $G_1 \land \dots \land G_n$, the upper limit for natural numbers} 
  \KwOut {Primitive conjunction $G_1 \land \dots \land G_m$ after simplification}
  $\mathcal{C}$ := $G_1 \land \dots \land G_n$ \;
  
  
  \ForEach {$G_i$ in $G_1 \land \dots \land G_n$}
  {
    \If{$G_i$ is of the form $T \in D$, and T is a ground term }
   {
      \If{T is in D}
      {
        Remove $G_i$ from $\mathcal{C}$\;
      }
      \Else
      {
      \Return \textbf{false}\;
      }
   }
   
   
   \If{$G_i$ is of the form $X \in D$, X is an arithmetic variable in $\mathcal{C}$, and D does not contain a number}
   {
     \Return \textbf{false}
   }
   
   \If{$G_i$ is of the form $T \in D$, T is an arithmetic term with at least one operation, and D does not contain a number}
   {
      \Return \textbf{false}\;
   }
  \If{$G_i$ is of the form  $t_1 \diamond t_2$ and both $t_1$ and $t_2$ are ground terms}
   {
     Let $G_i^\prime$ be obtained from $G_i$ where $<,\leq$ replaced with $\prec,\preceq$ respectively.
     \If {$G_i^\prime$ is true}
     {
        Remove $G_i$ from $\mathcal{C}$\;
     }
     \Else {
        \Return \textbf{false} \;
     }
   } 
   
   
    \If{$G_i$ is of the form  $t_1 \diamond t_2$, where one of $t_1$ and $t_2$ is an arithmetic term with at least one operation, number, arithmetic variable; and another one is a symbolic term (a string constant or a term built from a functional symbol),
    and diamond is either $=,<,\leq,\prec, or \preceq$ }
   {
        \Return \textbf{false} \;
   }
   
   \Return $\mathcal{C}$
  }  
\end{algorithm2e}



\begin{algorithm2e}[H]\caption{Split}
 \DontPrintSemicolon
  \KwIn{A clingcon rule of the form $r(X_1,\dots X_n):-BODY$} 
  \KwOut{A collection of clingcon rules $R$} 
  Let $BODY$ be $A_1,A_2,\dots A_m$.
  Let $\mathcal{G}$ be undirected graph with m nodes $N_1,\dots N_m$, such that there is an edge between $N_i$ and $N_j$ iff atoms $A_i$ and $A_j$ share a common variable.
  Let $C_1,\dots, C_k$ be connected components of $\mathcal{G}$. \;
  $R:=\emptyset$ \;
  \ForEach {connected component $C_i$ in   $C_1,\dots, C_k$} 
  {
    Let $C_i$ consists of nodes $N_{i_1},\dots N_{i_t}$.
    Let $X_1,\dots, X_p$ be all variables in $A_{i_1},\dots A_{i_t}$. \;
    $R:=R \cup r_i:-A_{i_1},\dots A_{i_t}$ \;
  } 
  $R:=R \cup r:-r_1,\dots,r_k.$ \;
  $R:=R \cup  :-not~r.$ \;
\end{algorithm2e}

  

\begin{algorithm2e}[H]\caption{Solve}
 \DontPrintSemicolon
  \KwIn{Primitive conjunction $G_1 \land \dots \land G_n$, the upper limit for natural numbers $maxint$.}
  \KwOut{\textbf{true} if $G_1 \land \dots \land G_n$ is satisfiable and \textbf{false} otherwise.}
   \tcc*[l]{Build rules for arithmetic constraints}
   $\Pi_{prolog}:= :-use\_module(library(clpfd)).$ \;
   $BODY:= {\bf true}$ \;


   \ForEach{primitive arithmetic $G_i$   of the form $t_1 \diamond t_2$}
   { 
     Replace $\prec, = , !=,<=$ with $\#<,\#=,\#=,\#=<$ respectively \;
     $BODY := BODY \land G_i$ \;
   }
   


   \ForEach{primitive arithmetic  $G_i$   of the form $(t\in D)$,where $D$ is of the form $[n1..n2]$}{
   $BODY := BODY \land  t~in~n_1..n_2$ \;
   }


   \ForEach{primitive arithmetic  $G_i$   of the form $\neg(t\in D)$,where $D$ is of the form $[n1..n2]$, and $g_i$ is an unique label for $G_i$}{
      $\Pi_{prolog} := \Pi_{prolog} \cup g_i(t) :- n\#>n_2. \cup g_i(t) :- n\#<n_2$.  
      $BODY := BODY \land  g_i(t)$ \;
   }


   \ForEach{primitive arithmetic $G_i$ of the form $(t \in D)$ and $\neg(t \in D)$ , where $D$ is not of the form $[n1..n2]$}
   {
      Remove all symbolic terms from D\; 
      Add the following rule to $\Pi_{prolog}$: (d is an unique label for D) \;
     \texttt{ set\_d(X):-member(X,[t1,\dots tn]).}\;  
     
   }
 
   \ForEach{primitive non-arithmetic $G_i$ of the form $\neg (t \in D)$:}
   {
    $BODY:=BODY \land \\+~set_d(t)$ \;
   }
   
  \ForEach{primitive non-arithmetic $G_i$ of the form $t \in D$:}
   {
    $BODY:=BODY \land set_d(t)$ \;
   }
   
   
   \ForEach{primitive non-arithmetic $G_i$ of the form $t_1 \diamond t_2$:}
   {
    $BODY:=BODY \land G_i$ \;
   }
 
    \ForEach {arithmetic variable $X$ in $BODY$} {
      $BODY:=integer(X) \land BODY$
    }
  
   
   Let $Y_1,\dots Y_n$ be the set of all variables in $BODY$ \;
   $\mathcal{R}:= p :-BODY$ \;
   
   $\Pi_{prolog}:=\Pi_{prolog} ~\cup~ \mathcal{R} $\;
   \If{$\Pi_{prolog}$ outputs 'yes' for query ?-p}
   {
     \Return \textbf{true} \;
   }
   \Else 
   {
     \Return \textbf{false}\;
   }
 \end{algorithm2e}
\end{document}

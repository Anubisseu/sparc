

\documentclass[12pt, letterpaper]{article}

\setlength{\topmargin}{-1.75cm} \setlength{\textheight}{22.5cm}
\setlength{\oddsidemargin}{0.25cm}
\setlength{\evensidemargin}{0.25cm} \setlength{\textwidth}{16.2cm}
\renewcommand{\figurename}{Table}
\usepackage{amssymb}
\usepackage{graphicx}
\usepackage{amsmath}
\usepackage[normalem]{ulem}
\usepackage{fontenc}
\usepackage{hyperref}
\usepackage{footnote}

\usepackage{palatino, url, multicol, listings} % for multiple columns
\lstset{mathescape=true, basicstyle=\ttfamily,}

%\usepackage{pictex}
%% in the .pictex output of xfig, there is command \colo
%% however the old version of pictex may not define this
%% so we define color here as empty
%\def \color#1]#2{}

\begin{document}

\newcommand{\hide}[1]{}
\newcommand{\exercise}[1]{}
\newcommand{\future}[1]{}
\newcommand{\otherquestions}[1]{}
\newcommand{\set}[1]{\{#1\}}
\newcommand{\pg}[1]{{\tt #1}}
\newtheorem{definition}{Definition}
\newcommand{\emptyclause}{\Box}
\def\st{\bigskip\noindent}
\newcommand{\lplus}
{
   \stackrel{+}{\gets}
}

\newcommand{\fe}[1] {
  \begin{frame}
    #1
  \end{frame}}

\newcommand{\eoa}{ {\bf End} of algorithm}

\newcommand{\ft}[1] {\frametitle{#1}}

\newcommand{\ie}[1] {
  \begin{itemize}
    #1
  \end{itemize}
}

\newcommand{\ee}[1] {
  \begin{enumerate}
    #1
  \end{enumerate}\label{marker}
}
\newcommand{\blk}[2] {
  \begin{block}{#1}
    #2
  \end{block}
}

\newtheorem{collorary}{Corollary}
\newtheorem{proposition}{Proposition}
\newtheorem{invariant}{Invariant}
\newtheorem{property}{Property}
\newtheorem{claim}{Claim}
\newtheorem{example}{Example}


\title{${\cal SPARC}$ manual}

\date{\today}
\maketitle
\section{Introduction}
A good knowledge representation methodology should allow one to:
\begin{itemize}
\item Identify and describe \emph{sorts} (types, kinds,
categories) of objects
populating a given domain.
\item Identify and precisely
define important \emph{properties} of these objects and
\emph{relationships} between them.
\end{itemize}
An \emph{abstract model} of the
domain will consist of sorts
and definitions produced by this process.
To deal with a particular problem one needs to
describe relevant objects and their sorts
and properties and use
the corresponding inference engine to solve the problem.
ASP based knowledge representation languages have powerful
means for describing non-trivial properties of objects
but lack the means of conveniently specifying objects
and their sorts as well as sorts of parameters
of the domain relations and functions. There were some
recent attempts to remedy the problem but we do not believe
that a sufficiently simple and powerful solution has been
found so far. In this manual we describe a knowledge representation
language ${\cal SPARC}$ which attempts to solve this problem and provide some examples of using the system we implemented.

\section{First Steps}
${\cal SPARC}$ system  works in two phases:
\begin{enumerate}
\item Translation of ${\cal SPARC}$ code to Answer Set Prolog (ASP) code.
\item Running ASP code on existing ASP solvers.
\end{enumerate}

\st So, to use the system you need two things:
\begin{enumerate}
\item The SPARC to ASP translator. It can be downloaded here: \url{https://github.com/iensen/sparc/blob/master/sparc.jar?raw=true}.
\item An ASP solver. We recommend using DLV for best compatibility with out translator. DLV can be downloaded here:
 \url{http://www.dlvsystem.com/dlv/#1}
You need to download \textit{static} version of the executable file.

\end{enumerate}
To demonstrate usage of the system, let's start with simple example.
\begin{verbatim}
sort definitions
person=$bob|tim|andy.
predicate declarations
teacher(person).
program rules
teacher(bob).
\end{verbatim}

\st This is a SPARC program. It consists of three sections:
\begin{itemize}
 \item Sort definitions. The section starts with the keywords \textit{sort definition} followed by definitions of the sorts in the program.
 \item Predicate declarations. The section starts with the keywords \textit{predicate declarations} followed by declarations of predicates of the program.
 \item Program rules. The sections starts with the keywords \textit{program rules}  and consists of a collection of rules satisfying ASP syntax.
\end{itemize}

\st To translate the program, run sparc.jar with arguments specifying the SPARC program and the file where the translation will put its result:
\begin{verbatim}
$ java -jar sparc.jar example.sp -o example.asp
SPARC to DLV translator V2.02
program translated
\end{verbatim}

\st To get the answer sets of the translated program, run DLV:

\begin{verbatim}
$ dlv example.asp
DLV [build BEN/Dec 21 2011   gcc 4.6.1]

{teacher(bob)}
\end{verbatim}

\st You should see that the answer set appeared. For a detailed description of ${\cal SPARC}$ system options, see 
\url{https://github.com/iensen/sparc/wiki/System-usage}.

\st For Linux, MacOS or Cygwin (Windows) users we prepared a bash script which combines translation and solver execution into one step.
It is available here: \url{https://github.com/iensen/sparc/raw/master/sparc}. Before using the script, make it executable by running command
\texttt{chmod +x sparc} from the command line.

The general syntax is :
\begin{verbatim}
 sparc input_file [dlv_options]
\end{verbatim}
For the complete list of dlv options, see 
\url{http://www.dlvsystem.com/html/DLV_User_Manual.html }

Here is an example of using the script:
\begin{verbatim}
$ sparc example.sp -filter=teacher
...
{teacher(bob)}
\end{verbatim}
\st To see more examples of ${\cal SPARC}$ programs, visit \url{https://github.com/iensen/sparc/wiki/Examples}.
\section{Detailed System Description}

\subsection{Sort definitions}



\st We define sorts as named collections of strings over a fixed alphabet $\Gamma$ consisting of:
\begin{itemize}
 \item latin letters: $\{a,b,c,d,..,z,A,B,C,D,..,Z\}$
 \item digits: $\{0,1,2,..,9\}$
 \item underscore: $\_$
\end{itemize}

\st We define string collections (languages) by means of expressions: regular expressions and set-theoretic operations, like \textit{union}, \textit{intersection} and \textit{difference}.
We denote the set of strings of language $L$ by $\mathcal{D}(L)$. 
In the following description, $\alpha$, $\beta$ and $\gamma$ denote characters from  $\Gamma$, $s$ denotes a string, $n$ , $n_1$ and $n_2$ 
denote non-negative numbers, $regexp$, $regexp_1$ and  $regexp_2$ denote regular  expressions.

\st In the table №1 we describe the syntax of \textit{regular expressions} used in the system.

\begin{figure}[h!]
  \caption{Regular Expressions}
\begin{minipage}{25cm} 
    \begin{tabular}{ | l | l | l | l |}
    \hline
    Syntax & Informal Description & Described language & Example \\ \hline
    $\alpha$ ($\alpha \in \Gamma$) & character & \{a\} & $a$  \\  \hline
    $regexp_1 regexp_2$ & concatenation & $\{s_1s_2: s_1 \in \mathcal{D}(regexp_1), s_2 \in \mathcal{D}(regexp_2)\}$ & $ab$ \\ \hline
    $regexp_1 | regexp_2$ & disjunction & $\mathcal{D}(regexp_1) \cup \mathcal{D}(regexp_2)$ & $a|b$ \\ \hline
    $\sim regexp$ & complement & $\{s: s\not\in \mathcal{D}(regexp) \}$ & $\sim a$ \\ \hline  
    $regexp\{n\}$ & n-repetition & $\{\underbrace{ss\ldots s}_\textrm{n times} : s\in \mathcal{D}(regexp) \}$ & $a\{10\}$ \\ \hline
    $regexp\{n,m\}$ & counting & $\mathcal{D}(regexp\{n\}) \cup \ldots \cup \mathcal{D}(regexp\{m\})$  & $a\{1,10\}$ \\ \hline
    $[\alpha - \beta]$ & range & $\{\gamma :  \alpha <=\gamma <= \beta \}$ & $[a-z]$ \\ \hline
    $(regexp)$ & parenthesis & $\mathcal{D}(regexp)$ & (a) \\ \hline
    \end{tabular}
 \end{minipage}
\end{figure}



\st Next we define expressions. In table №2,  {\em sort names} (denoted by $s,s_1,s_2,\ldots$)
and  {\em functional symbols}($f,f_1,f_2,\ldots$) 
are all  identifiers starting with a lowercase letter. {\em Variable names}(denoted by $V_1,V_2,\ldots$) 
are identifiers starting with a capital letter. $\$regexp$ denotes a language of all strings satisfying regular expression $regexp$.
$rel$ is a binary relation, one of $>=,<,>,<=,=$. $u,u_1,u_2, \ldots , u_n$ are strings over alphabet $\Gamma$.
$e,e_1,\ldots, e_n$ are expressions.


\begin{figure}[h!]
  \caption{Expressions.}
\begin{minipage}{25cm}
    \begin{tabular}{ | l | l | l | l |}
    \hline
    Syntax & Informal Description & Described language & Example \\ \hline
    $\$regexp$  & regular expression & $\mathcal{D}(regexp)$ & $\$[a-z]$  \\  \hline
    $n_1..n_2$ & integer range & $\{n: n_1<=n<=n_2$ & $1..4$ \\ \hline
    $s$ & sort name & \parbox{3cm}{$\{\mathcal{D}(e):$ $s$ was defined as $s=e\}$} & $s1$ \\ \hline
    \parbox{4cm}{$f(e_1[V_1]$ \footnote{Square brackets denote the optional part.}
    \footnote{The Current implementation allows variables only after expressions consisting of single sort name.}
    $,\ldots e_n[V_n])$ \\$[:\{V_i op V_j\}] $ } & functional term 
    & \parbox{3cm}{$\{f(u_1,\ldots, u_n) :$ \\ $u_1\in \mathcal{D}(e_1),$ \\ $\ldots u_n \in \mathcal{D}(e_n)$\\
     $[: u_i~op~u_j ]\}$ \footnote{The last condition $u_i$ op $u_j$ 
                          is present if and only if there is a condition
                          of the form $V_i$ op $V_j$ in expression.}}   
    & \parbox{2.3cm}{$put(block(X),$\\  $block(Y)):\{X!=Y\}$}\\ \hline
    $ e_1+e_2 $ & union & $\mathcal{D}(e_1)+\mathcal{D}(e_2)$ & $\$a+f(s)$ \\ \hline
    $ e_1*e_2 $ & intersection & $\mathcal{D}(e_1)*\mathcal{D}(e_2)$ & $\$a*f(s)$ \\ \hline 
    $ e_1-e_2 $ & difference & $\mathcal{D}(e_1) \backslash \mathcal{D}(e_2)$ & $\$a-f(s)$ \\ \hline
    $ (e) $     & parenthesis& $\mathcal{D}(e)$ & $(\$a+\$b)*s$ \\ \hline
    \end{tabular}
\end{minipage}
  
 \end{figure}
\st The $\$$ sign before a regular expression is used to avoid grammar ambiguity,
i.e., to make regular expression which are expressed by identifiers distinguishable from sort names. 
\pagebreak


\st \textbf{Example}
Consider the following sort definitions:
\begin{verbatim}
  s=$a|b.
  s1=s.
  s2=$s.
\end{verbatim}
\st Here sort s1 consists of string 'a' and 'b',  but s2 consist of only one string 's'.

\st The definition of sort named $s$ is an assignment of the form $s=e$,
  where  $e$ is an expression not containing sort names which were not defined in former definitions.
\section{Predicate Declarations}

\noindent  The second part of a  ${\cal SPARC}$ program starts with the keyword
\st
$predicate\ declarations$

\st and is followed by statements of the form

\st
$pred\_symbol(sortName,\dots,sortName)$

\st

Multiple declarations for one predicate symbol are not allowed.
\st For any sort name $SN$, the system includes declaration  $SN(SN)$ automatically.

\section{Program Rules}


\st The third part of a ${\cal SPARC}$ program consists of standard ASP rules and/or consistency restoring (cr)-rules.
\st
CR-rules are of the following form:

\begin{equation}
   [label:] l_0 \lplus l_1,  \ldots, l_k, not~l_{k+1} \ldots not~l_{n}
 \end{equation}
where $l$'s are literals.
Literals occurring in the heads of the rules must not be formed by predicate symbols
occurring as sort names in sort definitions.
\section{Directives}
Directives should be written before sort definitions, at the very beginning of a program.
${\cal SPARC}$ allows two types of directives:
\subsection{\#maxint}
Directive \#maxint specifies maximal nonnegative number which could be used in arithmetic calculations. For example,
\begin{verbatim}
 #maxint=15.
\end{verbatim}
\st limits integers to [0,15].
\subsection{\#const}
Directive \#const allows one to define constant values. The syntax is:
\begin{verbatim}
   #const constantName = constantValue.
\end{verbatim}      
\st where $constantName$  must begin with a lowercase letter and may be composed of letters, underscores and digits,
 and $constantValue$ is either a nonnegative number or the name of another constant defined above.  
\section{Answer Sets}
\noindent A set of ground literals $S$ is an {\em answer set} of a ${\cal SPARC}$ 
program $\Pi$ with regular rules only if $S$ is an answer set of an ASP program consisting of the same rules.

\st To define the semantics of a general ${\cal SPARC}$ program, we need notation for abductive support.
By $\alpha(r)$ we denote a regular rule
obtained from a consistency restoring rule $r$
by replacing $\lplus$ by $\leftarrow$;
$\alpha$ is expanded in the standard way to a set $X$ of CR-rules,
i.e., $\alpha(A) = \{\alpha(r)\; :\; r \in A\}$.
\st A %minimal (with respect to the preference relation $\leq$ of the program)
collection $A$ of CR-rules of $\Pi$ such that
\begin{enumerate}
\item $R \cup \alpha(X)$ is consistent (i.e., has an answer set), and
\item any $R_0$ satisfying the above condition has cardinality
which is greater than or equal to that of $R$
\end{enumerate}
is called an {\em abductive support} of $\Pi$.
\st A set of ground literals $S$ is an {\em answer set} of a ${\cal SPARC}$ program 
$\Pi$ if $S$ is an answer set of $R \cup \alpha(A)$, where $R$ is the set of regular rules of $\Pi$, for some abductive
support $A$ of $\Pi$.

\st \textbf{Example}
\begin{verbatim}
sort definitions
s1=$a.  % term "a" has sort "s1"
predicate declarations
p(s1).  %predicate  "p" accepts terms of sort s1 
q(s1).  %predicate  "q" accepts terms of sort s1 
program rules
p(a) :- not q(a).
-p(a).
q(a):+.  % this is a CR-RULE. 
\end{verbatim}
\st Result:
\begin{verbatim}
  ./sparc example2.sp
DLV [build BEN/Dec 21 2011   gcc 4.6.1]

SPARC to DLV translator V2.02
program translated
Best model: {-p(a), appl(r_0), q(a)}
Cost ([Weight:Level]): <[1:1]>

\end{verbatim}

\st Additional literal $appl(r_0)$ was added to the answer set, which means that the 
first cr-rule from the program was applied.
\section{Typechecking}
During typechecking  we perform a static check of each predicate from the program rules for 
satisfying type definitions provided in the first section.

\subsection{Type errors}
\begin{definition} (Matching)
A ground term $t$ \textit{matches} a set of ground literals $S$ if $t\in S$.
A term with variables $tv$  \textit{matches}  a ground literal $l$ if one of the following conditions holds:
\begin{enumerate}
 \item $tv$ is a variable.
 \item $tv$ is an arithmetic term (i.e, it consists of numbers, variables and arithmetic operations), and $l$ is a number.
 \item $tv$ is of the form $f_1(t_1,t_2,\ldots t_n)$, $l$ is of the form $f_2(u_1,u_2,\ldots u_n)$, $f_1=f_2$, and for all $1<=i<=n$ $t_i$ matches $u_i$.
\end{enumerate}
A term with variables $tv$  \textit{matches}  a set of  ground literals $S$ if it matches at least one of elements of $S$.
\end{definition}

We say that atom $p(t_1,t_2,\ldots,t_n)$  \textit{follows} sort definitions if the following conditions hold:
\begin{enumerate}
\item predicate $p$ was declared with exactly $n$ arguments, i.e., in predicate declarations' section of the program there is a declaration 
$p(s_1,\ldots,s_n)$, where $s_1,\ldots s_n$ are sort names.
\item for each term $1<=i<=n$ $t_i$ matches $\mathcal{D}(s_i)$. 
\end{enumerate}
For each atom $p(s_1,\ldots,s_n)$ type error is produced if it does not \textit{follow} sort definitions.

\subsection{Type warnings}
We say that sort $s$ is assigned to term $t$, if there is an atom $p(t_1,\ldots t_n)$,   which was declared as $p(s_1,\ldots s_n)$,and for some i $t=t_i$, $s=s_i$.
Type warning is produced for term $t$, if $\{s_1,\ldots s_n\}$ is a set of  sorts assigned to $t$, and $ {\cal D}(s_1) \cap \ldots {\cal D}(s_n)$ is empty.

\st \textbf{Example} 

Consider the following SPARC program:
\begin{verbatim}
 sort definitions
 s1=$1..4$ and s2=$5..15$ 
 predicate declarations
 p(s1).
 q(s2).  
 program rules
 p(X):-q(X)

\end{verbatim}
 For term $X$ (which is a variable in this case), sorts $s1$ and $s2$ are derived. 
 A warning is produced, because ${\cal D}(s_1) \cap \ldots {\cal D}(s_n)=\emptyset$.  
\begin{verbatim} 
%WARNING: Term X occurring in the rule p(X):-q(X).(line 8, column 2) has an empty grounding with respect to sort definitions
\end{verbatim}

\section{${\cal SPARC}$ and Unsafe Rules}

${\cal SPARC}$ helps to avoid \textit{unsafe rule} errors in most  cases.
\st \textbf{Example.}
\begin{verbatim}
sort definitions
s1=1..5.
predicate declarations
p(s1).
program rules
p(X). 
\end{verbatim}
The only rule in the program is unsafe (variable X does not occur in the body).
However, by translating the program and running ASP solver, we will be able to avoid this unsafety 
(with addition of auxiliary predicates).
Here is the execution trace:
\begin{verbatim}
$ ./sparc example3.sp -pfilter=p 
DLV [build BEN/Dec 21 2011   gcc 4.6.1]

SPARC to DLV translator V2.02
program translated
{p(1), p(2), p(3), p(4), p(5)}
\end{verbatim}
\section{${\cal SPARC}$ and ASPIDE}
TODO. 

\end{document}